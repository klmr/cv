\documentclass{klmr-cv}

\setotherlanguage{french}

\author{Konrad Rudolph}
\email{konrad.rudolph@gmail.com}
\website{klmr.me}

\newcommand*\csharp{C\#}
\newcommand*\cpp{C++}

\begin{document}

\maketitle

\begin{sidebar}
    \section{Degrees}

    \subsection{University of Cambridge\date{2016}}
    \entry{PhD}

    \subsection{Freie Universität Berlin\date{2011}}
    \entry{Master of Science}

    Bioinformatics

    \subsection{Freie Universität Berlin\date{2008}}
    \entry{Bachelor of Science}

    Major: bioinformatics

    Minor: physics

    \subsection{\textfrench{Cité Scolaire Internationale de Lyon}\date{2003}}
    \entry{Abitur \& baccalauréat}

    \section{Profiles}

    \subsection{GitHub}
    \href{http://github.com/klmr}{klmr}

    \subsection{ORCID}
    \href{http://orcid.org/0000-0002-9866-7051}{0000-0002-9866-7051}

    \subsection{Stack Overflow}
    \href{http://stackoverflow.com/users/1968/konrad-rudolph}{Konrad Rudolph}

    \section{Skills}

    \subsection{Languages}
    Fluency in written and spoken English, German \& French

    \subsection{Analysis}
    RNA-seq, ChIP-seq, motif analysis, data exploration

    \subsection{Programming}
    \cpp, C, R, Python, Perl, Bash, Make, \csharp, Java, PHP,~…

    \subsection{Tools}
    Version control, unit testing, reproducible research, literate programming,
    lexical/semantic analysis, Unix, Windows, macOS

    \subsection{Administrative experience}
    EMBL PhD Symposium organising committee
    EMBL Bioinformatics Workshop coordination
\end{sidebar}

\section{Research}

\subsection{EMBL-EBI, CRUK Cambridge Institute\date{2015--2016}}
\entry{Quantifying genome-wide expression data of repeat elements}

\subsection{EMBL-EBI\date{2014--2015}}
\entry{Does interaction between codon usage and tRNA abundance in mammals
    control translational efficiency?}

\subsection{EMBL-EBI\date{2011--2014}}
\entry{tRNA \& mRNA gene expression during mammalian development}

Group leader: Dr John Marioni

\subsection{Freie Universität Berlin\date{2010--2011}}
\entry{Generic parallelisation of a sequence analysis library}

Supervisor: Prof Dr Knut Reinert

\subsection{Freie Universität Berlin\date{2009}}
\entry{Implementation of a tool for T-invariant analysis in Petri nets of
    biochemical pathways}

Group leader: Prof Dr Ina Koch

\subsection{Freie Universität Berlin\date{2008}}
\entry{Implementation of a read mapping tool based on the pigeon-hole principle}

Supervisor: Prof Dr Knut Reinert

\section{Work}

\subsection{University of Cambridge\date{May 2016--}}
\entry{Postdoctoral research associate}

\subsection{EMBL-EBI\date{Oct 2015--Mar 2016}}
\entry{Postdoctoral fellow}

\subsection{EMBL-EBI\date{Oct 2011--Sep 2015}}
\entry{Predoctoral fellow}

\subsection{Independent consultancy\date{Jun 2011--Aug 2011}}
\entry{Developer}

Integration of FPGA kernel with \cpp{} library

\subsection{Illumina Inc.\date{Oct 2008--Jan 2009}}
\entry{Research associate (intern)}

Implementation of short-read mapping on GPGPUs with Nvidia CUDA

\subsection{Freie Universität Berlin\date{2008--2011}}
\entry{Tutor}

\subsection{ITosa\date{Jan 2007--Dec 2007}}
\entry{Full-stack developer}

\csharp{} Windows application; PHP/JavaScript/HTML web application \& database
frontend

\section{Teaching}

\subsection{EMBL-EBI\date{2011--2015}}
\entry{Bioinformatics workshop}

Bash scripting, Unix, Git, R, \LaTeX{}

\subsection{University of Cambridge\date{2017}}
\entry{smallRNA sequencing}

\subsection{University of Cambridge\date{2013--2015}}
\entry{Next generation sequencing course}

High-throughput sequencing techniques, with focus on ChIP-seq \& RNA-seq

\subsection{Freie Universität Berlin\date{2008--2011}}

Algorithms 101, Algorithms 102, Algorithms in bioinformatics, Database systems,
\cpp{}

\section{Selected publications}

% Bib(La)TeX does not cope at all with shared first authors (what the heck!).
% Hence we simply write the references here directly. 🤷  ¯\_(ツ)_/¯

\begin{enumerate}
    \item Bianca M Schmitt*, Konrad L M Rudolph*, Panagiota Karagianni,
        Nuno A Fonseca, Robert J White, Iannis Talianidis, Duncan T Odom,
        John C Marioni, Claudia Kutter. “High-resolution mapping of
        transcriptional dynamics across tissue development reveals a stable
        mRNA--tRNA interface.” \textit{Genome Research}, August 2014.
        \href{http://dx.doi.org/10.1101/gr.176784.114}{DOI:
        10.1101/gr.176784.114}.

    \item Konrad L M Rudolph*, Bianca M Schmitt*, Diego Villar, Robert J
        White, John C Marioni, Claudia Kutter, Duncan T Odom. “Codon-driven
        translation is stable across diverse mammalian cell states.”
        \textit{PLOS Genetics}, May 2016,
        \href{http://dx.doi.org/10.1371/journal.pgen.1006024}{DOI:
        10.1371/journal.pgen.1006024}.
\end{enumerate}

\section{Presentations}

\subsection{Invited talks}

Cambridge Epigenetics Symposium\date{2014}

\section{Scholarships}

EMBL PhD fellowship

\end{document}
