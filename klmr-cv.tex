\documentclass{klmr-cv}

\author{Konrad Rudolph}
\email{konrad.rudolph@gmail.com}
\website{careers.stackoverflow.com/klmr}

\newcommand*\csharp{C\#}
\newcommand*\cpp{C++}

\begin{document}

\maketitle

\section{Degrees}

\subsection{University of Cambridge}

\date{2011--now}
\item{\type{PhD}}
\item{Thesis “Investigating the link between tRNA and mRNA abundance in
    mammals”}
\item{Supervisor: Dr John Marioni}

\subsection{Freie Universität Berlin}

\date{2011}
\item{\type{Master of Science}}
\item{Thesis “Generic parallelisation of a sequence analysis library”}
\item{Supervisors: Prof Dr Knut Reinert, Dr Tim Conrad}

\subsection{Freie Universität Berlin}

\date{2008}
\item{\type{Bachelor of Science}}
\item{Thesis “Implementation of a read mapping tool based on the pigeon-hole
    principle”}
\item{Supervisors: Prof Dr Knut Reinert, Dr Gunnar Klau}
\item{Major in bioinformatics; minor in physics \& presentation skills}

\subsection{Cité Scolaire Internationale de Lyon}

\date{2003}
\item{\type{Abitur \& baccalauréat high school diploma}}

\section{Profiles}

\item{GitHub: \href{http://github.com/klmr}{klmr}}
\item{ORCID: \href{http://orcid.org/0000-0002-9866-7051}{0000-0002-9866-7051}}
\item{\href{http://stackoverflow.com/users/1968/konrad-rudolph}{Stack Overflow}}

\section{Research}

\date{2011--2014}
\item{tRNA \& mRNA gene expression during mammalian development}

\date{2014--2015}
\item{The interaction between codon usage and tRNA abundance in cancer}

\section{Teaching}

\subsection{Freie Universität Berlin}

\date{2008--2011}
\item{\type{Tutor}}
\item{Algorithms 101/102, Algorithms in bioinformatics, Databases 101, \cpp}

\subsection{EMBL-EBI}

\date{2011--2014}
\item{\type{Bioinformatics workshop}}
\item{Bash scripting, Unix, Git, \LaTeX}

\subsection{University of Cambridge}

\date{2013--2015}
\item{\type{Next generation sequencing course}}
\item{ChIP-seq, RNA-seq}

\section{Industry}

\subsection{Illumina, Inc.}

\date{Oct 2008--Feb 2009}
\item{\type{Research associate --- intern}}
\item{Research of the applicability of GPGPUs/Nvidia CUDA to the problem of
        short-read mapping}

\subsection{ITosa}

\date{Jan 2007--Jan 2008}
\item{\type{Front-end developer}}

\section{Publications}

% Bib(La)TeX does not cope at all with shared first authors. Hence we simply
% write the references here directly.

\begin{enumerate}
\listitem{Bianca M Schmitt, Konrad L M Rudolph, Panagiota Karagianni, Nuno A
    Fonseca, Robert J White, Iannis Talianidis, Duncan T Odom, John C Marioni
    \& Claudia Kutter. 2014. “High-resolution mapping of transcriptional
    dynamics across tissue development reveals a stable mRNA--tRNA interface.”
    \textit{Genome Research}, August.
    \href{http://dx.doi.org/10.1101/gr.176784.114}{doi:10.1101/gr.176784.114}.}
\listitem{Markus S Schröder, Dermot Harnett, Benedikt A Minke, Preethy
    Sasidharan Nair \& Committee Member Consortium. 2013. “Organizing a PhD
    Symposium --- an inside View.” \textit{EMBO Reports} 14 (10). EMBO Press:
    856--60.}
\end{enumerate}

\section{Presentations}

\subsection{Talks}

\begin{itemize}
    \listitem
        \date{2014}
        Cambridge Epigenetics Symposium
\end{itemize}

\subsection{Posters}

\begin{itemize}
    \listitem
        \date{2015}
        Biology of Genomes
    \listitem
        \date{2014}
        Genome Informatics
    \listitem
        \date{2013}
        Biology of Genomes
    \listitem
        \date{2009}
        German Conference on Bioinformatics
\end{itemize}

\section{Skills}

\begin{itemize}
    \listitem Fluency in written and spoken English, German \& French
    \listitem EMBI PhD Symposium organising committee
    \listitem EMBL Bioinformatics Workshop coordination
    \listitem Analysis (HTS, in particular RNA-seq, ChIP-seq, Motif analysis,
        data exploration)
    \listitem Programming (\cpp, C, R, Python, Perl, Bash, Make, \csharp, Java, PHP)
    \listitem Tools (Version control, reproducible research, literate
        programming, Parsers)
    \listitem Document preparation (\LaTeX)
\end{itemize}

\end{document}
